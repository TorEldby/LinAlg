\documentclass[a4]{article}

\usepackage{hyperref}

\title{Notes on Linear Algebra lib}
\author{Tor Eldby}
\title{}

\begin{document}
\maketitle

\section{Introduction}
I want this document to be a rough description of the features I want for this library

\subsection{Sources}
There are a few sources I wish to have listed. Obviously there are math books, but I also want to have a set of "example libraries" or projects others have made. I am admittedly not well versed in library creation, and want to use this project as a good way to get better at it.\\
\textbf{Books}
\begin{itemize}
\item Linear Algebra and its Applications
\end{itemize}

\textbf{Links}
\begin{itemize}
\item Writing a C++ linear algebra library from scratch\footnote{\url{https://sowb.github.io/2021/09/15/writing-a-c-linear-algebra-library-from-scratch.}}
\end{itemize}

\section{Basics}
The base feature I want is to create matrices and vectors, along with handling basic mathematical operations between matrices and vectors (Addition, subtraction, scalar multiplication, determinants)

\subsection{Data representation}
I believe it should be a class. \\
Should the data be templated? \\
Should the dimensions be given through templates? \\
How should the data be stored? Vectors? Arrays? \\
Should the data be a 1-dimensional flattened array, or some 2D std::vector?\\ 
Should the data be a set of library vectors ala column-vectors?

\subsection{Addition}
Note about vector and matrix addition, dimensionality, etc.

\subsection{Subtraction}
Note about vector and matrix subtraction, dimensionality, etc.

\subsection{Scalar multiplication}
Scalar multiplication of both vectors and matrices simply apply the scalar multiplication to each entry in the vector or matrix


\section{Advanced}
More advanced features would be reduction algorithms, calculating determinants

\section{Use cases}
Any application of linear algebra, really. It would be fun for this to be useful in my OpenGL exploration, or to use it as a linear algebra library for numerical simulations

\end{document}